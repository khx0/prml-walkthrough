\documentclass[11pt,DINA4, fleqn]{amsart}
\pagestyle{empty}
\newcommand{\R}{\Bbb{R}}
\newtheorem{thm}{Frage}
\usepackage{amsmath,amssymb,amsthm,amsfonts}

\usepackage{wasysym}
\usepackage[ngerman]{babel}
\usepackage[T1]{fontenc} 
\usepackage[utf8]{inputenc}
\usepackage{lmodern}

\usepackage{geometry}
\geometry{hmargin=2.5cm,vmargin={2cm,1cm}}

\usepackage{blindtext}
\usepackage{multirow,booktabs}

\usepackage{pifont}
\usepackage{dsfont}
\usepackage{enumitem}
%\setitemize{leftmargin=0.5cm} 
%\setlength{\itemindent}{1in}
\usepackage{fancybox,framed}

\usepackage{listings} \lstset{numbers=left, numberstyle=\tiny, numbersep=5pt} \lstset{language=c++} 

\usepackage{tcolorbox}

\definecolor{mycolor}{rgb}{0.122, 0.435, 0.698}

\usepackage{xcolor}
\definecolor{darkgreen}{rgb}{0.14,0.72,0.31}
\definecolor{MyBoxColor}{rgb}{0.9,0.9,0.9}
\newenvironment{shadedSmaller}{
  \def\FrameCommand{\fboxsep=\FrameSep \colorbox{MyBoxColor}}
  \MakeFramed {\advance\hsize-2\width\FrameRestore}}
{\endMakeFramed}

\newtcolorbox{mybox_tc2}[1]{colback=red!5!white,colframe=red!75!black,fonttitle=\bfseries,title=#1}

\newtcolorbox{mybox_tc3}[1]{colback=darkgreen!5!white,colframe=darkgreen!75!black,fonttitle=\bfseries,title=#1}


\newenvironment{shadedSmallerPadding}{
  \def\FrameCommand{\fboxsep=0.15cm \colorbox{MyBoxColor}}
  \MakeFramed {\advance\hsize-1.1\width\FrameRestore}}
{\endMakeFramed}

\usepackage{faktor}

\usepackage{tikz}
\usetikzlibrary{matrix,arrows}

%----------------------------------------------------------------------------------------------------------------

\def\vw{\boldsymbol{w}\xspace}



%%%%%%%%%%%%%%%%%%%%%%%%%%%%%%%
\newcommand{\Ldot}{\overset{\textbf{.}\kern0.23em}{\mathbf{L}}}
%%%%%%%%%%%%%%%%%%%%%%%%%%%%%%%



\begin{document}



%\begin{tikzpicture}[normal line/.style={->},font=\scriptsize] \path[normal line] (m-1-1) edge (m-1-2); %\end{tikzpicture}

\tikzset{node distance=2cm, auto}




%                                      $\mathds{Z}/n\mathds{Z}$

\begin{flushleft}
{\sc \Large PRML Exercise 1.1 \& 1.2} \hfill \today \\
\medskip
Nikolas Schnellbächer \underline{\hspace{6.53in}} \\
\end{flushleft}

\section*{Exercise 1.1 -- Polynomial Least Squares Curve Fitting}
This exercise considers polynomial least squares curve fitting. 
For this purpose we consider the sum of squares error function as defined in equation 1.2 of chapter 1, which we repeat below
\begin{align}
E(\vw) = \dfrac{1}{2} \sum \displaylimits_{n = 1}^{N}
\biggl(y(x_n, \vw) - t_n)\biggl)^2 \, .
\label{eq:error1}
\end{align}
Here we consider the model function $y$ to be a polynomial of degree $M$ in $x$ with coefficients $\vw = (w_0, w_1, w_2, \dots , w_M)^{T}$, such that
\begin{align}
y(x,\vw) = w_0 + w_1 x + w_2 x^2 + \dotsc + w_M x^M = \sum \displaylimits_{j = 0}^{M} w_j x^j
\label{eq:polynom}
\end{align}
is the polynomial model function (also known as fit function) with $M + 1$ summands, containing $M+1$ parameters $w_j$. These parameters are also often referred to as weights.

\begin{mybox_tc3}{Exercise 1.1 -- Polynomial Least Squares Curve Fitting}
Consider the sum-of-squares error function given in equation \eqref{eq:error1}
with a polynomial model function $y(x,\vw)$ of order $M$ as specified in equation \eqref{eq:polynom}.
Given a set of $N$ datapoints $(x_n, t_n)$ with $n = 1, \dotsc, N$,
 derive the linear system, which minimizes this
error function, \textit{i.e.} show how to find the optimal solution for $\vw$
which minimizes the sum-of-squares error
\begin{align}
\vw = \text{argmin}\biggl[ E(\vw)\biggl] \, ,
\end{align}
given the $N$ data points.
\end{mybox_tc3}

\section{Event-driven stochastic simulation algorithms}
At any given instant in the event-driven Gillespie algorithm, the system is in a well defined state at a given reference time $t_0$. According to the rules of the physical system that you consider, there may be several reaction channels, each with an associated propensity (probability per unit time) $a_r$. The probability per unit time, that any reaction at all fires, is then simply given by the sum of all propensities
\begin{align}
a_0 = \sum_r a_r\, ,
\end{align} 
where $r$ sums over all reaction channels.
Hence the next event time in this algorithm is thus at each instant of the algorithm sampled from an exponential distribution with parameter $\lambda = a_0$. Also note that the reaction channel propensities usually change in each iteration of the algorithm, such that at every new instant one considers a new Poisson process with a new Poisson rate $a_0$ according to the current propensities, and samples the next event time as the first-arrival time from this Poisson process at this instant. Using the inversion method this next arrival time is then obtained by generating a uniform random number $r$ in the unit interval and applying the following transformation
\begin{align}
\tau = -\ln(r) / a_0 \quad \text{where} \quad r \sim U[0,1].
\end{align}
Having started at an arbitrary reference time $t_0$, the time of the next event is then accordingly $t_0+\tau$.

In our first example of this event-driven algorithm we simulated the Poisson process itself with a Poisson rate $\alpha$. In this particular example there is only a single reaction channel, which has a constant propensity $\alpha$, so $a_0 = \alpha$ in this case. So here the reaction propensity is not state-dependent which simplifies matters a lot. Hence one does not need to reevaluate these propensities anew in every single iteration and simply needs to sample next event times in each iteration.

\section{Generalization}

Additionally I would like to mention, that the result, of the probability distribution of the first arrival time being exponentially distributed is a special case for $k=1$ of the more general probability distribution of the k-th arrival time of a Poisson process.
\begin{mybox_tc3}{\emph{$k$-th arrival time distribution} of the Poisson process}
	The $k$-th arrival time distribution of a Poisson process with Poisson rate $\lambda$ is
	\begin{align}
f_{T_k}(t) = \frac{\lambda^k t^{k-1}e^{-\lambda t}}{(k-1)!}.
	\end{align}
\end{mybox_tc3}

\end{document}









