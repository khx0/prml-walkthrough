\documentclass[11pt,DINA4, fleqn]{amsart}
\pagestyle{empty}
\newcommand{\R}{\Bbb{R}}
\newtheorem{thm}{Frage}
\usepackage{amsmath,amssymb,amsthm,amsfonts}

\usepackage{wasysym}
\usepackage[ngerman]{babel}
\usepackage[T1]{fontenc} 
\usepackage[utf8]{inputenc}
\usepackage{lmodern}

\usepackage{geometry}
\geometry{hmargin=2.5cm,vmargin={2cm,1cm}}

\usepackage{blindtext}
\usepackage{multirow,booktabs}

\usepackage{pifont}
\usepackage{dsfont}
\usepackage{enumitem}
%\setitemize{leftmargin=0.5cm} 
%\setlength{\itemindent}{1in}
\usepackage{fancybox,framed}

\usepackage{listings} \lstset{numbers=left, numberstyle=\tiny, numbersep=5pt} \lstset{language=c++} 

\usepackage{tcolorbox}

\definecolor{mycolor}{rgb}{0.122, 0.435, 0.698}

\usepackage{xcolor}
\definecolor{darkgreen}{rgb}{0.14,0.72,0.31}
\definecolor{MyBoxColor}{rgb}{0.9,0.9,0.9}
\newenvironment{shadedSmaller}{
  \def\FrameCommand{\fboxsep=\FrameSep \colorbox{MyBoxColor}}
  \MakeFramed {\advance\hsize-2\width\FrameRestore}}
{\endMakeFramed}

\newtcolorbox{mybox_tc2}[1]{colback=red!5!white,colframe=red!75!black,fonttitle=\bfseries,title=#1}

\newtcolorbox{mybox_tc3}[1]{colback=darkgreen!5!white,colframe=darkgreen!75!black,fonttitle=\bfseries,title=#1}


\newenvironment{shadedSmallerPadding}{
  \def\FrameCommand{\fboxsep=0.15cm \colorbox{MyBoxColor}}
  \MakeFramed {\advance\hsize-1.1\width\FrameRestore}}
{\endMakeFramed}

\usepackage{faktor}

\usepackage{tikz}
\usetikzlibrary{matrix,arrows}

%----------------------------------------------------------------------------------------------------------------

\def\vw{\boldsymbol{w}\xspace}
\def\vb{\boldsymbol{b}\xspace}
\def\vy{\boldsymbol{y}\xspace}
\def\mA{\boldsymbol{A}\xspace}
\def\mV{\boldsymbol{V}\xspace}

%%%%%%%%%%%%%%%%%%%%%%%%%%%%%%%
\newcommand{\Ldot}{\overset{\textbf{.}\kern0.23em}{\mathbf{L}}}
%%%%%%%%%%%%%%%%%%%%%%%%%%%%%%%



\begin{document}



%\begin{tikzpicture}[normal line/.style={->},font=\scriptsize] \path[normal line] (m-1-1) edge (m-1-2); %\end{tikzpicture}

\tikzset{node distance=2cm, auto}




%                                      $\mathds{Z}/n\mathds{Z}$

\begin{flushleft}
{\sc \Large PRML Exercise 1.1 \& 1.2} \hfill \today \\
\medskip
Nikolas Schnellbächer \underline{\hspace{6.53in}} \\
\end{flushleft}

\section*{Exercise 1.1 -- Polynomial Least Squares Curve Fitting}
This exercise considers polynomial least squares curve fitting. 
For this purpose we consider the sum of squares error function as defined in equation 1.2 of chapter 1, which we repeat below
\begin{align}
E(\vw) = \dfrac{1}{2} \sum \displaylimits_{n = 1}^{N}
\biggl(y(x_n, \vw) - t_n)\biggl)^2 \, .
\label{eq:error1}
\end{align}
Here we consider the model function $y$ to be a polynomial of degree $M$ in $x$ with coefficients $\vw = (w_0, w_1, w_2, \dots , w_M)^{T}$, such that
\begin{align}
y(x,\vw) = w_0 + w_1 x + w_2 x^2 + \dotsc + w_M x^M = \sum \displaylimits_{j = 0}^{M} w_j x^j
\label{eq:polynom}
\end{align}
is the polynomial model function (also known as fit function) with $M + 1$ summands, containing $M+1$ parameters $w_j$. These parameters are also often referred to as weights.

\begin{mybox_tc3}{Exercise 1.1 -- Polynomial Least Squares Curve Fitting}
Consider the sum-of-squares error function given in equation \eqref{eq:error1}
with a polynomial model function $y(x,\vw)$ of order $M$ as specified in equation \eqref{eq:polynom}.
Given a set of $N$ data points $(x_n, t_n)$ with $n = 1, \dotsc, N$,
 derive the linear system, which minimizes this
error function, \textit{i.e.} show how to find the optimal solution $\vw^*$,
which minimizes the sum-of-squares error
\begin{align}
\vw^{*} =  \underset{\vw}{\text{argmin}}\biggl[ E(\vw)\biggl] \, ,
\end{align}
given the $N$ data points $(x_n, t_n)$.
\end{mybox_tc3}

\subsection*{Solution}
To find the optimal solution $\vw^*$ which minimizes the energy functional, we take the partial derivatives of $E(\vw)$ with respect to the weight parameters $w_j$ for all $j$ and set them to zero, to find the extremal solution. This leads to a set of $M+1$ equations, which happen to be linear in the coefficients $w_j$. Thus we can solve polynomial least squares fitting by solving a simple linear system.

\begin{align}
\dfrac{\partial E}{\partial w_0} &= \sum \displaylimits_{n = 1}^{N}
\biggl(\underbrace{w_0 + w_1 x_n + w_2 x_n^2 + \dotsc + w_M x_n^M}_{=y(x_n,\vw)} - t_n\biggl) \cdot 1
\overset{!}{=} 0 \\
\dfrac{\partial E}{\partial w_1} &= \sum \displaylimits_{n = 1}^{N}
\biggl(w_0 + w_1 x_n + w_2 x_n^2 + \dotsc + w_M x_n^M - t_n\biggl) \cdot x_n
\overset{!}{=} 0 \\
\phantom{\dfrac{\partial E}{\partial w_M}}&\,\,\,\vdots \\
\dfrac{\partial E}{\partial w_M} &= \sum \displaylimits_{n = 1}^{N}
\biggl(w_0 + w_1 x_n + w_2 x_n^2 + \dotsc + w_M x_n^M - t_n\biggl) \cdot x_n^M
\overset{!}{=} 0
\end{align}
Rearranging all $M+1$ equations by dragging the $n$ summation over all terms, we can identically rewrite this set of equations as
\begin{align}
w_0 N + w_1 \sum\displaylimits_{n = 1}^{N}x_n
+ w_2 \sum\displaylimits_{n = 1}^{N}x_n^2 + \dotsc + w_M \sum\displaylimits_{n = 1}^{N}x_n^M  &= \sum\displaylimits_{n = 1}^{N} t_n \\
w_0 \sum\displaylimits_{n = 1}^{N} x_n + w_1 \sum\displaylimits_{n = 1}^{N}x_n^2
+ w_2 \sum\displaylimits_{n = 1}^{N}x_n^3 + \dotsc + w_M \sum\displaylimits_{n = 1}^{N}x_n^{M + 1}  &= \sum\displaylimits_{n = 1}^{N} x_n t_n
 \\
\phantom{\dfrac{\partial E}{\partial w_M}}&\,\,\,\vdots \\
w_0 \sum\displaylimits_{n = 1}^{N} x_n^M + w_1 \sum\displaylimits_{n = 1}^{N}x_n^{M+1}
+ w_2 \sum\displaylimits_{n = 1}^{N}x_n^{M+2} + \dotsc + w_M \sum\displaylimits_{n = 1}^{N}x_n^{2M}  &= \sum\displaylimits_{n = 1}^{N} x_n^M t_n
\end{align}
This is a set of $M+1$ linear equations in $w_0, w_1, w_2, \dotsc, w_M$. We can immediately rewrite this in matrix form and find.
\begin{align}
\underbrace{
\begin{pmatrix}
N & \sum\displaylimits_{n = 1}^{N}x_n & \sum\displaylimits_{n = 1}^{N}x_n^2 & \dots & \sum\displaylimits_{n = 1}^{N}x_n^M \\
\sum\displaylimits_{n = 1}^{N}x_n & \sum\displaylimits_{n = 1}^{N}x_n^2 &
\sum\displaylimits_{n = 1}^{N}x_n^3 & \dots &
\sum\displaylimits_{n = 1}^{N}x_n^{M + 1} \\
\vdots & \vdots & \vdots && \vdots \\ 
\sum\displaylimits_{n = 1}^{N}x_n^M &
\sum\displaylimits_{n = 1}^{N}x_n^{M+1} &
\sum\displaylimits_{n = 1}^{N}x_n^{M+2} &
\dotsc &  \sum\displaylimits_{n = 1}^{N}x_n^{2M}
\end{pmatrix}}_{=\mA} \cdot
\underbrace{\begin{pmatrix}
w_0 \\ w_1 \\ \vdots \\ w_M
\end{pmatrix}}_{=\vw}
=
\underbrace{\begin{pmatrix}
\sum\displaylimits_{n = 1}^{N} t_n \\
\sum\displaylimits_{n = 1}^{N} x_n t_n \\
\vdots \\
\sum\displaylimits_{n = 1}^{N} x_n^M t_n 
\end{pmatrix}}_{=\vb} \, \in \mathbb{R}^{M+1}
\label{eq:explicitMatrixForm1}
\end{align}
By solving the linear system $\mA \cdot \vw = \vb$ for the weights $\vw$ thus gives us the optimal solution, which minimizes the sum of squared errors.

\begin{mybox_tc3}{Polynomial Least Squares Curve Fitting}
	For $N$ given data points $(x_n, t_n)$ and a polynomial fitting model of order $M$,
	the optimal solution $\vw^*$ to the polynomial least squares curve fitting problem is the solution of the linear system
	\begin{align}
	\mA \cdot \vw = \vb \quad \Longleftrightarrow \quad \sum \displaylimits_{j = 0}^{M} A_{ij} \cdot w_j = b_i  \qquad \forall\quad i \in \{0, 1, 2, \dotsc, M\}
	\end{align}
	where
	\begin{align}
	A_{ij} &= \sum \displaylimits_{n = 1}^{N} (x_n)^{i + j} \\
	(\vw)_j &= w_j \\
	b_i &= \sum \displaylimits_{n = 1}^{N} (x_n)^{i} \, t_n
	\end{align}
	Both indices $i$ and $j$ run from $0, 1, 2, \dotsc, M$.
	The matrix $\mA$ is obviously symmetric by construction $A_{ij} = A_{ji}$.
\end{mybox_tc3}

\section*{Exercise 1.2 -- Polynomial least squares curve fitting with quadratic regularization}
We can regularize the sum of squares error function from equation \eqref{eq:error1} by adding a corresponding regularization term. 

\begin{align}
\tilde{E}(\vw) = \dfrac{1}{2} \sum \displaylimits_{n = 1}^{N}
\biggl(y(x_n, \vw) - t_n)\biggl)^2 + \dfrac{\lambda}{2} \, || \vw ||^2 \, .
\label{eq:error2}
\end{align}
Here we have
\begin{align}
||\vw||^2 = \vw^T \cdot \vw  = w_0^2 + w_1^2 + w_2^2 + \dotsc +  w_M^2 = \sum \displaylimits_{j = 0}^{M} w_j^2
\end{align}
as the sum of squared weight coefficients. The global coefficient $\lambda$ controls the strength between the regularization term compared to the sum-of-squares error term.

\begin{mybox_tc3}{Exercise 1.2 -- Polynomial least squares curve fitting with quadratic regularization}
	Derive the coupled set of linear equations (analogous to exercise 1.1) satisfied by the coefficients
	$w_i$ which minimize the regularized sum-of-squares error function given by equation \eqref{eq:error2}, given the $N$ data points $(x_n, t_n)$, $n = 1, 2, \dotsc, N$.
\end{mybox_tc3}

\subsection*{Solution}
To find the optimal solution $\vw^*$ which minimizes the energy functional $\tilde{E}$, we again take the partial derivatives of $\tilde{E}(\vw)$ with respect to the weight parameters $w_j$ for all $j$ as before.
Setting these equations to zero again leads to a system of coupled equations which for this error functional again happens to be linear in the coefficients $w_j$.

We start by taking the derivative with respect to $w_0$.
\begin{align}
\dfrac{\partial \tilde{E}}{\partial w_0} &= \dfrac{1}{2}\sum \displaylimits_{n = 1}^{N}
2 \biggl(\underbrace{w_0 + w_1 x_n + w_2 x_n^2 + \dotsc + w_M x_n^M}_{=y(x_n,\vw)} - t_n\biggl) \cdot 1
+ \dfrac{\lambda}{2} \cdot 2 w_0 \\
&=\sum \displaylimits_{n = 1}^{N} \biggl(y(x_n,\vw) - t_n\biggl) + \lambda w_0 \overset{!}{=} 0
\end{align}
Next, moving the $n$ summation inwards we can rewrite this equation as
\begin{align}
w_0 (N + \lambda) + w_1 \sum \displaylimits_{n = 1}^{N} x_n 
+ w_2 \sum \displaylimits_{n = 1}^{N} x_n^2 + \dotsc 
+ w_M \sum \displaylimits_{n = 1}^{N} x_n^M = \sum \displaylimits_{n = 1}^{N} t_n \, ,
\end{align}
putting all coefficient on the left hand side and keeping the constant terms on the right hand side.
Next we treat the derivative with respect to $w_1$.
\begin{align}
\dfrac{\partial \tilde{E}}{\partial w_1} &= \dfrac{1}{2}\sum \displaylimits_{n = 1}^{N}
2 \biggl(\underbrace{w_0 + w_1 x_n + w_2 x_n^2 + \dotsc + w_M x_n^M}_{=y(x_n,\vw)} - t_n\biggl) \cdot \, x_n
+ \dfrac{\lambda}{2} \cdot 2 w_1 \\
&=\sum \displaylimits_{n = 1}^{N} \biggl(y(x_n,\vw) - t_n\biggl) \, x_n + \lambda w_1 \overset{!}{=} 0
\end{align}
By moving the sum inwards we rearrange the equation to yield
\begin{align}
w_0 \sum \displaylimits_{n = 1}^{N} x_n 
+ w_1 \left( \lambda + \sum \displaylimits_{n = 1}^{N} x_n^2 \right) 
+ w_2 \sum \displaylimits_{n = 1}^{N} x_n^3 + \dotsc 
+ w_M \sum \displaylimits_{n = 1}^{N} x_n^{M + 1} = \sum \displaylimits_{n = 1}^{N} x_n \, t_n \, .
\end{align}
We repeat this routine up to the final $M+1$-th equation which considers the derivative with respect to the coefficient $w_M$.
\begin{align}
\dfrac{\partial \tilde{E}}{\partial w_M} &= \dfrac{1}{2}\sum \displaylimits_{n = 1}^{N}
2 \biggl(\underbrace{w_0 + w_1 x_n + w_2 x_n^2 + \dotsc + w_M x_n^M}_{=y(x_n,\vw)} - t_n\biggl) \cdot \, x_n^M
+ \dfrac{\lambda}{2} \cdot 2 w_M \\
&=\sum \displaylimits_{n = 1}^{N} \biggl(y(x_n,\vw) - t_n\biggl) \, x_n^M + \lambda w_M \overset{!}{=} 0
\end{align}
Moving the sum inwards gives
\begin{align}
w_0 \sum \displaylimits_{n = 1}^{N} x_n^M 
+ w_1 \sum \displaylimits_{n = 1}^{N} x_n^{M + 1} 
+ w_2 \sum \displaylimits_{n = 1}^{N} x_n^{M + 2} 
+ \dotsc 
+ w_M \left( \lambda + \sum \displaylimits_{n = 1}^{N} x_n^{2M} \right)
 = \sum \displaylimits_{n = 1}^{N} x_n^M \, t_n \, .
\end{align}
Below we restate the full coupled system consisting of $(M+1)$ equations.
\begin{align}
w_0 (N + \lambda) + w_1 \sum \displaylimits_{n = 1}^{N} x_n 
+ w_2 \sum \displaylimits_{n = 1}^{N} x_n^2 + \dotsc 
+ w_M \sum \displaylimits_{n = 1}^{N} x_n^M &= \sum \displaylimits_{n = 1}^{N} t_n \\
w_0 \sum \displaylimits_{n = 1}^{N} x_n 
+ w_1 \left( \lambda + \sum \displaylimits_{n = 1}^{N} x_n^2 \right) 
+ w_2 \sum \displaylimits_{n = 1}^{N} x_n^3 + \dotsc 
+ w_M \sum \displaylimits_{n = 1}^{N} x_n^{M + 1} &= \sum \displaylimits_{n = 1}^{N} x_n \, t_n \\
&\vdots \\
w_0 \sum \displaylimits_{n = 1}^{N} x_n^M 
+ w_1 \sum \displaylimits_{n = 1}^{N} x_n^{M + 1} 
+ w_2 \sum \displaylimits_{n = 1}^{N} x_n^{M + 2} 
+ \dotsc 
+ w_M \left( \lambda + \sum \displaylimits_{n = 1}^{N} x_n^{2M} \right)
&= \sum \displaylimits_{n = 1}^{N} x_n^M \, t_n \, .
\end{align}
Now we write this in matrix form very similar to equation \eqref{eq:explicitMatrixForm1} above.
\begin{align}
\underbrace{
	\begin{pmatrix}
	N + \lambda& \sum\displaylimits_{n = 1}^{N}x_n & \sum\displaylimits_{n = 1}^{N}x_n^2 & \dots & \sum\displaylimits_{n = 1}^{N}x_n^M \\
	\sum\displaylimits_{n = 1}^{N}x_n & \lambda + \sum\displaylimits_{n = 1}^{N}x_n^2 &
	\sum\displaylimits_{n = 1}^{N}x_n^3 & \dots &
	\sum\displaylimits_{n = 1}^{N}x_n^{M + 1} \\
	\vdots & \vdots & \vdots && \vdots \\ 
	\sum\displaylimits_{n = 1}^{N}x_n^M &
	\sum\displaylimits_{n = 1}^{N}x_n^{M+1} &
	\sum\displaylimits_{n = 1}^{N}x_n^{M+2} &
	\dotsc &  \lambda + \sum\displaylimits_{n = 1}^{N}x_n^{2M}
	\end{pmatrix}}_{=\tilde{\mA}} \cdot
\underbrace{\begin{pmatrix}
	w_0 \\ w_1 \\ \vdots \\ w_M
	\end{pmatrix}}_{=\vw}
=
\underbrace{\begin{pmatrix}
	\sum\displaylimits_{n = 1}^{N} t_n \\
	\sum\displaylimits_{n = 1}^{N} x_n t_n \\
	\vdots \\
	\sum\displaylimits_{n = 1}^{N} x_n^M t_n 
	\end{pmatrix}}_{=\vb} \, \in \mathbb{R}^{M+1}
\label{eq:explicitMatrixForm1}
\end{align}
Note that $\vw$ and $\vb$ are unchanged and only the matrix is modified by adding the regularization parameter $\lambda$ on the diagonal, when comparing this to the result without regularization.
\begin{align}
\tilde{\mA} = \mA  + \lambda \mathds{1}
\end{align}

\begin{mybox_tc3}{Polynomial least squares curve fitting with quadratic regression}
	For $N$ given data points $(x_n, t_n)$ and a polynomial fitting model of order $M$,
	the optimal solution $\vw^*$ to the polynomial least squares curve fitting problem with quadratic regularization is the solution of the linear system
	\begin{align}
	\tilde{\mA} \cdot \vw = \vb \quad \Longleftrightarrow \quad \sum \displaylimits_{j = 0}^{M} \tilde{A}_{ij} \cdot w_j = b_i  \qquad \forall\quad i \in \{0, 1, 2, \dotsc, M\}
	\end{align}
	or equivalently
	\begin{align}
	(\mA + \lambda \mathds{1}) \cdot \vw = \vb \qquad \text{where} \quad \tilde{\mA} = \mA + \lambda \mathds{1} \, .
	\end{align}
	Here $\lambda$ is the regularization parameter, $\mathds{1}$ is an $(M+1) \times (M+1)$ identity matrix and the linear system is given by
	\begin{align}
	A_{ij} &= \sum \displaylimits_{n = 1}^{N} (x_n)^{i + j} \\
	(\vw)_j &= w_j \\
	b_i &= \sum \displaylimits_{n = 1}^{N} (x_n)^{i} \, t_n
	\end{align}
	Both indices $i$ and $j$ run from $0, 1, 2, \dotsc, M$.
	The matrix $\mA$ is obviously symmetric by construction $A_{ij} = A_{ji}$.
	Solving the problem of polynomial least square curve fitting with quadratic regularization is hence identical to the least squares problem without regularization, up to the global regularization parameter $\lambda$ on the diagonal of the system matrix. 
\end{mybox_tc3}

Making again the connection to the Vandermonde matrix $\mV$ we can equivalently write down the linear system as
\begin{align}
(\mA + \lambda \mathds{1}) \vw &=  \vb \quad \Longleftrightarrow \\
(\mV^T \mV + \lambda \mathds{1}) \vw &=  \vb
\end{align}
where we have
\begin{align}
\tilde{\mA} = A + \lambda \mathds{1} = \mV^T\mV + \lambda \mathds{1}
\end{align}
and
\begin{align}
\vb = \mV^T \vy \, .
\end{align}

\end{document}









