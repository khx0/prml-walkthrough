\documentclass[11pt,DINA4, fleqn]{amsart}
\pagestyle{empty}
\newcommand{\R}{\Bbb{R}}
\newtheorem{thm}{Frage}
\usepackage{amsmath,amssymb,amsthm,amsfonts}

\usepackage{wasysym}
\usepackage[ngerman]{babel}
\usepackage[T1]{fontenc} 
\usepackage[utf8]{inputenc}
\usepackage{lmodern}

\usepackage{geometry}
\geometry{hmargin=2.5cm,vmargin={2cm,1cm}}

\usepackage{blindtext}
\usepackage{multirow,booktabs}

\usepackage{pifont}
\usepackage{dsfont}
\usepackage{enumitem}
%\setitemize{leftmargin=0.5cm} 
%\setlength{\itemindent}{1in}
\usepackage{fancybox,framed}

\usepackage{listings} \lstset{numbers=left, numberstyle=\tiny, numbersep=5pt} \lstset{language=c++} 

\usepackage{tcolorbox}

\definecolor{mycolor}{rgb}{0.122, 0.435, 0.698}

\usepackage{xcolor}
\definecolor{darkgreen}{rgb}{0.14,0.72,0.31}
\definecolor{MyBoxColor}{rgb}{0.9,0.9,0.9}
\newenvironment{shadedSmaller}{
  \def\FrameCommand{\fboxsep=\FrameSep \colorbox{MyBoxColor}}
  \MakeFramed {\advance\hsize-2\width\FrameRestore}}
{\endMakeFramed}

\newtcolorbox{mybox_tc2}[1]{colback=red!5!white,colframe=red!75!black,fonttitle=\bfseries,title=#1}

\newtcolorbox{mybox_tc3}[1]{colback=darkgreen!5!white,colframe=darkgreen!75!black,fonttitle=\bfseries,title=#1}


\newenvironment{shadedSmallerPadding}{
  \def\FrameCommand{\fboxsep=0.15cm \colorbox{MyBoxColor}}
  \MakeFramed {\advance\hsize-1.1\width\FrameRestore}}
{\endMakeFramed}

\usepackage{faktor}

\usepackage{tikz}
\usetikzlibrary{matrix,arrows}

%----------------------------------------------------------------------------------------------------------------

\def\vw{\boldsymbol{w}\xspace}
\def\vb{\boldsymbol{b}\xspace}
\def\vy{\boldsymbol{y}\xspace}
\def\mA{\boldsymbol{A}\xspace}
\def\mV{\boldsymbol{V}\xspace}

%%%%%%%%%%%%%%%%%%%%%%%%%%%%%%%
\newcommand{\Ldot}{\overset{\textbf{.}\kern0.23em}{\mathbf{L}}}
%%%%%%%%%%%%%%%%%%%%%%%%%%%%%%%

\begin{document}

%\begin{tikzpicture}[normal line/.style={->},font=\scriptsize] \path[normal line] (m-1-1) edge (m-1-2); %\end{tikzpicture}

\tikzset{node distance = 2cm, auto}


%                                      $\mathds{Z}/n\mathds{Z}$

\begin{flushleft}
{\sc \Large PRML Exercise 1.12} \hfill \today \\
\medskip
Nikolas Schnellbächer \underline{\hspace{6.53in}} \\
\end{flushleft}

\section*{Exercise 1.12 -- Expectation value for the mean and variance maximum likelihood estimators of a univariate Gaussian}
We consider a univariate Gaussian distribution with probability density
\begin{align}
\mathcal{N}\left(x \,\bigl| \, \mu, \sigma^2\right) = \dfrac{1}{\sqrt{2\pi \sigma^2}}\exp\left(-\dfrac{(x-\mu)^2}{2\sigma^2}\right) \, .
\end{align}
A common inference problem is to guess this distribution's parameters $\mu$ and $\sigma^2$ based on a finite sample, drawn form this distribution, \textit{i.e.}
inferring the distribution based on $N$ sample points $x_1, \dots, x_N$.

\begin{mybox_tc3}{Exercise 1.12 -- Expectation values of ML estimators}
The \emph{maximum likelihood} (ML) estimators for the mean and the variance of a univariate Gaussian distribution are
\begin{align}
\mu_{\text{ML}} &= \dfrac{1}{N} \sum\displaylimits_{n = 1}^{N} x_n \\
\sigma^2_{\text{ML}} &= \dfrac{1}{N} \sum\displaylimits_{n = 1}^{n}\bigl(x_n - \mu_{\text{ML}}\bigl)^2 \, .
\end{align}
Show that the expectation values of these two maximum likelihood estimators are given by
\begin{align}
\mathbb{E}\bigl[\mu_{\text{ML}}\bigl] &= \mu \\
\mathbb{E}\bigl[\sigma^2_{\text{ML}}\bigl] &= \left(\frac{N - 1}{N}\right) \sigma^2
\end{align}
As an intermediate result, start by showing that
\begin{align}
\mathbb{E}\bigl[x_n x_m\bigl] = \mu^2 + \delta_{nm} \sigma^2 \, ,
\label{eq:interResult}
\end{align}
where $x_n$ and $x_m$ are two independent and identically distributed (i.i.d.) univariate random variables $x_n, x_m \sim \mathcal{N}\left(x \,\bigl| \, \mu, \sigma^2\right)$ and $\delta_{nm}$ is the Kronecker delta.
It is defined as
\begin{align}
\delta_{nm} = \begin{cases}
1 & \text{for} \, n = m \\
0 & \text{for} \, n \neq m \, .
\end{cases}
\end{align}
\end{mybox_tc3}

\subsection*{Solution}

We start by showing the intermediate result as stated in equation \eqref{eq:interResult}. When $n\neq m$, we have that
\begin{align}
\mathbb{E}\bigl[x_n x_m\bigl] &=
\int \displaylimits_{-\infty}^{\infty}\int \displaylimits_{-\infty}^{\infty}
x_n x_m \, 
\mathcal{N}\left(x_n \,\bigl| \, \mu, \sigma^2\right) \,
\mathcal{N}\left(x_m \,\bigl| \, \mu, \sigma^2\right) \, dx_n dx_m \\
&= 
\left( \, \,
\underbrace{
\int \displaylimits_{-\infty}^{\infty}
x_n \,
\mathcal{N}\left(x_n \,\bigl| \, \mu, \sigma^2\right) \, \, dx_n
}_{=\mathbb{E}[x_n] = \mu}\right)^2
= \mu^2 \, ,
\end{align}
where we used the simple fact, that the first moment of a Gaussian distribution is $\mathbb{E}[x_n] = \mu$.
If now $n=m$, we have
\begin{align}
\mathbb{E}\bigl[x_n x_m\bigl] &= \mathbb{E}\bigl[x_n^2\bigl] = \int \displaylimits_{-\infty}^{\infty}
x_n^2 \, 
\mathcal{N}\left(x_n \,\bigl| \, \mu, \sigma^2\right) \, dx_n =
\mu^2 + \sigma^2 \,
\label{eq:2ndmoment}
\end{align}
where the last step reduces to a standard exponential integral.
In summary we find
\begin{align}
\mathbb{E}\bigl[x_n x_m\bigl] 
 = \mu^2 + \delta_{nm} \, \sigma^2
 =\begin{cases}
 \mu^2 & \text{for} \quad n\neq m \\s
 \mu^2 + \sigma^2 & \text{for} \quad n = m
 \end{cases}
\end{align}
With this intermediate result at hand, we can tackle the initial problem \textit{i.e.} to determine the expectation values of the \emph{maximum likelihood estimators} for the mean and the variance of a univariate \emph{Gaussian}. \\

\noindent
\textbf{Expectation value of the ML estimator for the mean}\\
\begin{align}
\mathbb{E}\bigl[\mu_{\text{ML}}\bigl] = 
\mathbb{E}\left[\dfrac{1}{N} \, \sum\displaylimits_{n=1}^{N} x_n\right] = \dfrac{1}{N} \sum \displaylimits_{n=1}^{N} \, \mathbb{E}\bigl[x_n\bigl] = \dfrac{1}{N} \cdot N \mu = \mu
\end{align}
This simple result (using the linearity of the mean) shows that the expectation value of the ML based estimator for the mean ($\mu_{\text{ML}}$) on average coincides with the true mean $\mu$ of the distribution.\\

\noindent
\textbf{Expectation value of the ML estimator for the variance}\\
For the expectation value of the ML estimator for the variance we need to do a little more work.
\begin{align}
\mathbb{E}\left[\sigma^2_{\text{ML}}\right] &= 
\mathbb{E}\left[\dfrac{1}{N} \, \sum\displaylimits_{n=1}^{N} \left( x_n - \mu_{\text{ML}}\right)^2\right] \\
&= \dfrac{1}{N} \sum \displaylimits_{n=1}^{N} \, \mathbb{E}\left[ \left(x_n - \mu_{\text{ML}}\right)^2\right] \\
&= \dfrac{1}{N} \sum \displaylimits_{n=1}^{N}
\biggl(
\mathbb{E}\bigl[x_n^2\bigl] - 2 \mathbb{E}\bigl[x_n\mu_{\text{ML}}\bigl]
+ \mathbb{E}\bigl[\mu_{\text{ML}}^2\bigl]
\biggl)
\label{eq:3summands}
\end{align}
In the last expression we have three terms in the round brackets, which we will derive term by term.
\begin{itemize}
	
	\item
	The first summand is the second moment of the distribution, as shown before in equation \eqref{eq:2ndmoment}. It is a simple standard exponential integral, and yields (as before)
	\begin{align}
	\mathbb{E}\bigl[x_n^2\bigl] = \mu^2 + \sigma^2 \, .
	\end{align}
	
	\item The second summand is a little more work.
	\begin{align}
	\mathbb{E}\bigl[x_n \mu_{\text{ML}}\bigl] = \mathbb{E}\left[
	x_n \dfrac{1}{N}\sum\displaylimits_{i=1}^{N} x_i
	\right] &=  \dfrac{1}{N} \sum\displaylimits_{i=1}^{N} \mathbb{E}\bigl[x_n x_i\bigl] \\
	&=
	\dfrac{1}{N} \sum\displaylimits_{i=1}^{N}\biggl(\mu^2 + \delta_{ni} \, \sigma ^2 \biggl) \\
	&= \mu^2 + \dfrac{1}{N} \sum\displaylimits_{i=1}^{N} \delta_{ni} \, \sigma^2 \\
	&= \mu^2 + \dfrac{1}{N} \, \sigma^2 \,
	\end{align}
	where we used out intermediate result form equation \eqref{eq:interResult}, stating that $\mathbb{E}[x_n x_i] = \mu^2 + \delta_{ni} \, \sigma^2$.
	
	\item The last (third) summand from equation \eqref{eq:3summands} computes to the following result.
	\begin{align}
	\mathbb{E}\bigl[\mu_{\text{ML}}^2\bigl] &= \mathbb{E} \left[
	\left(\dfrac{1}{N} \sum\displaylimits_{i=1}^{N} x_i\right)
	\left(\dfrac{1}{N} \sum\displaylimits_{j=1}^{N} x_j\right)
	\right] \qquad \text{linearity of the mean}\\
	&= \dfrac{1}{N^2} \sum\displaylimits_{i=1}^{N}\sum\displaylimits_{j=1}^{N}
	\mathbb{E}\bigl[x_i x_j\bigl] \\
	&= \dfrac{1}{N^2} \sum\displaylimits_{i=1}^{N}\sum\displaylimits_{j=1}^{N}
	\biggl(\mu^2 + \delta_{ij} \sigma^2\biggl) \\
	&= \dfrac{1}{N^2}\, N^2\mu^2 + \dfrac{1}{N^2} \, N\sigma^2 \\
	&= \mu^2 + \dfrac{1}{N} \sigma^2
	\end{align}
	
\end{itemize}
Now we can put everything back together.
\begin{align}
\mathbb{E}\left[\sigma^2_{\text{ML}}\right] 
&= \dfrac{1}{N} \sum \displaylimits_{n=1}^{N}
\biggl(
\mathbb{E}\bigl[x_n^2\bigl] - 2 \mathbb{E}\bigl[x_n\mu_{\text{ML}}\bigl]
+ \mathbb{E}\bigl[\mu_{\text{ML}}^2\bigl]
\biggl) \\
&= \dfrac{1}{N} \sum\displaylimits_{n=1}^{N}
\left(
\mu^2 + \sigma^2 - 2\left(\mu^2 + \dfrac{1}{N}\sigma^2\right) + \mu^2 + \dfrac{1}{N} \sigma^2
\right) \\
&=\sigma^2 - 2 \dfrac{1}{N}\sigma^2 + \dfrac{1}{N} \sigma^2 = \sigma^2 \left(1 - \dfrac{1}{N}\right) \\
&= \left(\dfrac{N-1}{N}\right) \, \sigma^2
\end{align}

\begin{mybox_tc3}{Expectation values of ML estimators}
This concludes this exercise and we have found the expectation values for the \emph{maximum likelihood estimators} of the mean and the variance of a univariate \emph{Gaussian} distribution.
\begin{align}
\mathbb{E}\bigl[\mu_{\text{ML}}\bigl] &= \mu \\
\mathbb{E}\bigl[\sigma^2_{\text{ML}}\bigl] &= \left(\frac{N - 1}{N}\right) \sigma^2
\end{align}
This result is quite interesting, since we learn the following from it. When we sample from a Gaussian distribution and wish to infer the distribution's parameters, using the \emph{maximum likelihood} estimators will (on average) reproduce an accurate estimator for the mean ($\mu_{\text{ML}}= \mu$), but systematically underestimate the true variance $\sigma^2$. This is important to realize. This underestimation effect cannot be compensated by more samples, but is a systematic weakness of the \emph{maximum likelihood} based estimators.
Since
\begin{align}
\dfrac{N-1}{N} < 1 \, ,
\end{align}
for all $N > 0$ we have a systematic bias for the variance estimator.
\end{mybox_tc3}

\end{document}
