%%%%%%%%%%%%%%%%%%%%%%%%%%%%%%%%%%%%%%%%%%%%%%%%%%%%%%%%%%%%%%%%%%%%%%%%%
% author: Nikolas Schnellbaecher
% contact: khx0@posteo.net
% file: main.tex
% date: 2020-05-08
%%%%%%%%%%%%%%%%%%%%%%%%%%%%%%%%%%%%%%%%%%%%%%%%%%%%%%%%%%%%%%%%%%%%%%%%%

\documentclass[11pt, DINA4, fleqn]{amsart}

\usepackage[T1]{fontenc} 
\usepackage[utf8]{inputenc}

\usepackage[english]{babel}
\usepackage{lmodern}

% math packages
\usepackage{amsmath, amssymb, amsthm, amsfonts}
\usepackage{bm}

\usepackage{wasysym}
\usepackage{blindtext}
\usepackage{multirow}
\usepackage{booktabs}
\usepackage{pifont}
\usepackage{dsfont}
\usepackage{enumitem}
\usepackage{fancybox}
\usepackage{framed}
%\usepackage{colortbl}

\usepackage{tabularx}
\usepackage{multirow}

\usepackage{listings} 
\lstset{numbers = left, numberstyle = \tiny, numbersep = 5pt} 
\lstset{language = c++} 

\usepackage{geometry}
\geometry{hmargin = 2.5cm,
		  vmargin = {2cm, 3.0cm},
	  	  footskip = 1.0cm}




% color packages
\usepackage{color}
\usepackage{graphicx}
\usepackage[table]{xcolor}
%\usepackage[table]{xcolor}
\usepackage{tcolorbox}

\definecolor{mycolor}{rgb}{0.122, 0.435, 0.698}
\definecolor{darkgreen}{rgb}{0.14, 0.72, 0.31}
\definecolor{MyBoxColor}{rgb}{0.9,0.9,0.9}
% \definecolor{CiteBlue}{RGB}{28, 58, 189}
\definecolor{CiteBlue}{RGB}{31, 119, 180} % mpl C0 color

\newenvironment{shadedSmaller}{
  \def\FrameCommand{\fboxsep=\FrameSep \colorbox{MyBoxColor}}
  \MakeFramed {\advance\hsize-2\width\FrameRestore}}
{\endMakeFramed}

\newtcolorbox{mybox_tc2}[1]{colback=red!5!white,colframe=red!75!black,fonttitle=\bfseries,title=#1}

\newtcolorbox{mybox_tc3}[1]{colback=darkgreen!5!white,colframe=darkgreen!75!black,fonttitle=\bfseries,title=#1}

\newenvironment{shadedSmallerPadding}{
  \def\FrameCommand{\fboxsep=0.15cm \colorbox{MyBoxColor}}
  \MakeFramed {\advance\hsize-1.1\width\FrameRestore}}
{\endMakeFramed}

\usepackage{fancyhdr}

\setlength{\headheight}{15pt}


\pagestyle{fancy}
\fancyhf{}
\fancyhead{}
\fancyfoot[RO]{\sffamily\bfseries\vrule width 0.8pt\hskip1mm\thepage}
\fancyfoot[LE]{\sffamily\bfseries\thepage \hskip1mm \vrule width 0.8pt}

\fancypagestyle{plain}{%
	\renewcommand{\headrulewidth}{0pt}%
	\fancyhf{}%
	\fancyfoot[LE,RO]{\sffamily\bfseries\vrule width 0.8pt\hskip1mm\thepage}%
}

\renewcommand{\headrulewidth}{0pt}

\usepackage{faktor}


% load hyperref package as last package
\usepackage{hyperref} 

\hypersetup{
	colorlinks,
	linkcolor= CiteBlue,
	citecolor = CiteBlue,
	urlcolor = CiteBlue
}

%%%%%%%%%%%%%%%%%%%%%%%%%%%%%%%%%%%%%%%%%%%%%%%%%%%%%%%%%%%%%%%%%%%%%%%%%
% custom definitions
\def\vw{\boldsymbol{w}\xspace}
\def\vb{\boldsymbol{b}\xspace}
\def\vy{\boldsymbol{y}\xspace}
\def\mA{\boldsymbol{A}\xspace}
\def\mV{\boldsymbol{V}\xspace}
%%%%%%%%%%%%%%%%%%%%%%%%%%%%%%%%%%%%%%%%%%%%%%%%%%%%%%%%%%%%%%%%%%%%%%%%%

\begin{document}

\pagestyle{fancy}

\begin{flushleft}
{\sc \Large Supplemental Notes on Decision Theory} \hfill \today \\
\medskip
Nikolas Schnellbächer \underline{\hspace{6.53in}} \\
\end{flushleft}

\section{ad figure 1.24 till figure 1.27}
These notes cover some clarifying remarks about the part on decision theory in chapter 1 and the accompanying figures 1.24, 1.26 and 1.27. In the covered explanatory example the problem considered consists of two random variables,
a continuous input variable $X$ on a domain $x\in\mathcal{X}$ and a discrete random variable $\mathcal{C}$ with two states $C_k\in \{C_1, C_2 \}$. The decision problem is to classify if a given input $x$ is either of the two binary states.

\subsection{Supplemental remarks on figure 1.24}
The plot in figure 1.24 displays two curves for $p(x \, | \, C_1)$ and $p(x \, | \, C_2)$. First of all make sure, that you appreciate, that these two curves combined
form the full joint probability distribution $p(x, \, C)$ of this problem. And as usual every quantity can be derived from having full information about the underlying joint distribution. Although the axes in figure 1.24 do not reveal any scale, it is useful to reiterate that each of these curves $p(x, \, C_k)$ for $k\in\{1, 2\}$ individually is not normalized but only the full joint is normalized, such that
\begin{align}
\mathcal{N} = \sum_{k=1}^{2} \, \, \int\displaylimits_{\mathcal{X}} \, p(x, C_k) \, dx = 1
\end{align}
is satisfied. The integral of the two shown curved individually is simply a standard marginalization and yields
\begin{align}
\int\displaylimits_{\mathcal{X}} p(x, C_1) \, dx = p(C_1) \quad  \text{and} \quad
\int\displaylimits_{\mathcal{X}} p(x, C_2) \, dx = p(C_2).
\end{align}
The knowledge about $p(C_1)$ and $p(C_2)$ is, in this simple example, of course the full marginalized distribution $p(C)$, which is itself of course also properly normalized, such that
\begin{align}
\sum_{k=1}^{2} p(C_k) = p(C_1) + p(C_2) = 1
\end{align}
holds.

The marginalization to obtain $p(x)$ from the joint is here also trivial and simply obtained by summing
\begin{align}
p(x) = \sum_{k=1}^{2} p(x, C_k) = p(x, C_1) + p(x, C_2) \, ,
\label{eq:marginal_x}
\end{align}
which is once again a properly normalized distribution of $x$ itself, such that
\begin{align}
\int\displaylimits_{\mathcal{X}} p(x) \, dx = 1.
\end{align}
Long story short: Having the joint we can immediately get the marginalized distributions and check that both the joint and (both) marginalized distributions are indeed all normalized.

\subsection{Supplemental remarks on figure 1.26}
In Figure 1.26 the chapter discusses the posterior conditional distributions $p(C_k \, | \, x)$ as a function of $x$.
Remember, this is the probability for the class $C_k$ given that the input variable takes the value $x$. Viewed as a function of $x$ we can actually straight forward write down how to obtain this conditional distribution, starting from the full joint distribution $p(x, C)$ as shown in figure 1.24.
For this consider having both $p(x, C_1)$ and $p(x, C_2)$ available. For a given input value $x$, the probability of belonging to class $C_1$ is then simply given by the value of $p(x, C_1)$ at this $x$ value, weighted by the total statistical weight of both classes $p(x, C_1) + p(x,C_2)$ at this value $x$,
\textit{i.e.} we arrive at 
\begin{align}
p(C_1 | x) = \dfrac{p(x, C_1)}{p(x, C_1) + p(x, C_2)} = \dfrac{p(x, C_1)}{p(x)} \, ,
\end{align}
which we could rewrite using the marginal distribution $p(x)$ as stated in equation \eqref{eq:marginal_x}.





\newpage


We consider a set of training data $\{\boldsymbol{\mathsf{x}}, \boldsymbol{\mathsf{t}}\}$, where
\begin{align}
\boldsymbol{\mathsf{x}} &= (x_1, \dots, x_N)^T \, , \\
\boldsymbol{\mathsf{t}} &= (t_1, \dots, t_N)^T
\end{align}
are the input variables $x_n$ and their corresponding target values $t_n$, respectively.
The central idea is using probability distribution to  express our uncertainty over the value of the target variable $t$, \textit{i.e.} we state
\begin{align}
p(t \, | \, x, \bm{w}, \beta) = \mathcal{N}\left(
t \, \bigl| \, y(x,\bm{w}), \beta^{-1}\right) \, .
\label{eq:tUncertainty}
\end{align}
The parameter $\beta$ is called \emph{precision} and correspond to the inverse variance of the distribution.

In curve fitting, we use the existing training data $\{\boldsymbol{\mathsf{x}}, \boldsymbol{\mathsf{t}}\}$ to infer the unknown parameters $\bm{w}$ and $\beta$.
We will below show how this works using a \emph{Maximum Likelihood} ansatz.
We assume that the data is drawn independently from the distribution \eqref{eq:tUncertainty}. Then the likelihood function is given by
\begin{align}
p(\boldsymbol{\mathsf{t}} \, | \, \boldsymbol{\mathsf{x}}, \bm{w}, \beta) = 
\prod\displaylimits_{n = 1}^{N}
\mathcal{N}\left(
t_n \, \bigl| \, y(x_n,\bm{w}), \beta^{-1}\right)
\label{eq:likelihoodFunc}
\end{align}
Instead of maximizing the likelihood function directly it is convenient to maximize its logarithm.
\begin{align}
\ln\biggl(p(\boldsymbol{\mathsf{t}} \, | \, \boldsymbol{\mathsf{x}}, \bm{w}, \beta)\biggl) &= \ln\left(\,
\prod\displaylimits_{n=1}^{N} \dfrac{1}{\sqrt{2\pi \beta^{-1}}} \, \exp\left[
-\dfrac{\beta}{2}\biggl(t_n - y(x_n,\bm{w})\biggl)^2
\right]
\right) \\
&=
\sum\displaylimits_{n=1}^{N}
\left[
\ln\left(\dfrac{1}{\sqrt{2\pi \beta^{-1}}}\right) -
\dfrac{\beta}{2}\biggl(y(x_n, \bm{w}) - t_n\biggl)^2 \,
\right] \\
&=
-\dfrac{\beta}{2}\sum\displaylimits_{n=1}^{N} \biggl(y(x_n, \bm{w}) - t_n\biggl)^2 +
N \ln\left(\left(\dfrac{\beta}{2\pi}\right)^{1/2}\right) \\
&= 
-\dfrac{\beta}{2}\sum\displaylimits_{n=1}^{N} \biggl(y(x_n, \bm{w}) - t_n\biggl)^2 +
\dfrac{N}{2}\ln(\beta)
- \dfrac{N}{2}\ln(2\pi)
\label{eq:logLikelihood}
\end{align}
We start by considering the inferred solution for the parameter weights $\bm{w}$ according to the maximum likelihood principle and will call this solution $\bm{w}_{\text{ML}}$. For this we need to maximize the log likelihood from equation \eqref{eq:logLikelihood} w.r.t to $\bm{w}$.
Since the last two terms do not depend on $\bm{w}$ we can omit them from this consideration. Second, scaling the log likelihood by a positive constant coefficient does not alter the solution of this maximization, \textit{i.e.} we can replace the $\beta/ 2$ in front of the summation by simply $1/2$. And finally, instead of maximizing the log likelihood, we can equivalently minimize the negative log likelihood. In summary the function that we now seek to minimize reads
\begin{align}
\dfrac{1}{2} \sum\displaylimits_{n=1}^{N} \biggl(y(x_n, \bm{w}) - t_n\biggl)^2 =: E(\bm{w}) \, ,
\end{align}
which reduces to the well known sum-of-squares error function $E(\bm{w})$.
This means that maximizing the likelihood is equivalent, so far as we are concerned with determining $\bm{w}$, to minimizing the \emph{sum-of-squares} error function.
Or in other words: The \emph{sum-of-squares} error function arises naturally in the context of maximum likelihood under the assumption of a Gaussian noise model and statistically independent data samples. We find $\bm{w}$ as before by solving the corresponding linear system (see equation \eqref{eq:linearSystem} or the previous sections on least squares curve fitting).

We can then equally use the maximum likelihood ansatz to determine the precision parameter $\beta_{\text{ML}}$. The derivative of equation \eqref{eq:logLikelihood} w.r.t to $\beta$ yields
\begin{align}
\dfrac{\partial}{\partial \beta} \ln\biggl(p(\boldsymbol{\mathsf{t}} \, | \, \boldsymbol{\mathsf{x}}, \bm{w}, \beta)\biggl) &= -\dfrac{1}{2}\sum\displaylimits_{n=1}^{N} \biggl(y(x_n, \bm{w}) - t_n\biggl)^2
+\,\, \dfrac{N}{2\beta} \stackrel{!}{=} 0 \, .
\end{align}
Solving the last expression for $\beta$, we find
\begin{align}
\beta_{\text{ML}} = \dfrac{N}{\sum\displaylimits_{n=1}^{N} \bigl(y(x_n, \bm{w}_{\text{ML}}) - t_n\bigl)^2} \, ,
\end{align}
or equivalently
\begin{align}
\dfrac{1}{\beta_{\text{ML}}} = \dfrac{1}{N} \sum\displaylimits_{n=1}^{N} \biggl(y(x_n, \bm{w}_{\text{ML}}) - t_n\biggl)^2 \, .
\end{align}
In this maximum likelihood ansatz, we first determine $\bm{w}_{\text{ML}}$ and use it subsequently for the determination of $\beta_{\text{ML}}$.

\begin{mybox_tc3}{Probabilistic curve fitting - maximum likelihood approach}
We consider dataset $\{\boldsymbol{\mathsf{x}}, \boldsymbol{\mathsf{t}}\}$
and a (polynomial) model function $y(x,\bm{w})$ and
assume that
\begin{itemize}
	\item the data points $\{(x_n, t_n)\}$ are statistically independent
	\item and that the target values $t_n$ are distributed according to
	\begin{align}
	t_n  \sim p(t_n \, | \, x_n, \bm{w}, \beta) = \mathcal{N}\left(
	t_n \, \bigl| \, y(x_n,\bm{w}), \beta^{-1}\right) 
	\end{align}
	a Gaussian distribution with mean $\mu = y(x_n, \bm{w})$ and variance $\sigma^2 = \beta^{-1}$.
\end{itemize}
	Then maximizing the likelihood (log likelihood), is equivalent to minimizing the \emph{sum-of-squares} error function
	\begin{align}
	E(\bm{w}) =\dfrac{1}{2} \sum\displaylimits_{n=1}^{N} \biggl(y(x_n, \bm{w}) - t_n\biggl)^2.
	\end{align}
The solution for the model parameters $\bm{w}_{\text{ML}}$ and $\beta_{\text{ML}}$ are then for a polynomial model function $y(x,\bm{w})$ given by
solving the linear system
\begin{align}
\mA \cdot \vw = \vb
\label{eq:linearSystem}
\end{align}
where
\begin{align}
A_{ij} = \sum \displaylimits_{n = 1}^{N} (x_n)^{i + j} \, , \quad
(\vw)_j = w_j \quad \text{and} \quad
b_i = \sum \displaylimits_{n = 1}^{N} (x_n)^{i} \, t_n \, ,
\end{align}
where both indices $i$ and $j$ run from $0, 1, 2, \dotsc, M$.
With the solution $\bm{w}_{\text{ML}}$, the maximum likelihood solution for the precision parameter $\beta_{\text{ML}}$ reads
\begin{align}
\dfrac{1}{\beta_{\text{ML}}} = \sigma^2_{\text{ML}}= \dfrac{1}{N} \sum\displaylimits_{n=1}^{N} \biggl(y(x_n, \bm{w}_{\text{ML}}) - t_n\biggl)^2
\end{align}
\end{mybox_tc3}

%%%%%%%%%%%%%%%%%%%%%%%%%%%%%%%%%%%%%%%%%%%%%%%%%%%%%%%%%%%%%%%%%%%%%%%%%%%%%%%%
%\begin{figure}[h]
%	\centering
%	\includegraphics[width=0.42\textwidth]{figure_1_16.pdf}
%	\caption{Schematic illustration of the uncertainty of the target variable $t$ in form of a Gaussian conditional distribution as specified in equation \eqref{eq:tUncertainty}, for which the mean is given by the polynomial function $y(x,\bm{w})$ and the precision is set by the parameter $\beta = 1/\sigma^2$.
%	\label{fig:figure_1_16}}
%\end{figure}
%%%%%%%%%%%%%%%%%%%%%%%%%%%%%%%%%%%%%%%%%%%%%%%%%%%%%%%%%%%%%%%%%%%%%%%%%%%%%%%%

\section{Bayesian polynomial curve fitting}

\subsection{Errata}
On page 31 there is a typo. The expression
$\bm{\phi}(x_n)\bm{\phi}(x)^T$ in equation (1.72) should read $\bm{\phi}(x_n)\bm{\phi}(x_n)^T$.

\subsection{Additional remarks on the Bayesian polynomial curve fitting results}
\begin{itemize}
	\item What is plotted in figure 1.17 (Bishop book) is essentially a confidence interval
	\begin{align}
	m(x) \pm \text{var}(x)^{1/2} \, .
	\end{align}
	
	\item ad equation (1.70): This equation in the original text reads
	\begin{align}
	m(x) = \beta \bm{\phi}(x)^T \, \underbrace{\bm{S} \, \sum\displaylimits_{n=1}^{N}\bm{\phi}(x_n)t_n}_{=:\bm{\xi}} \, .
	\label{eq:bayesCurveFitMean}
	\end{align}
	We will have a closer look at the expression
	\begin{align}
	\bm{S} \, \underbrace{\sum\displaylimits_{n=1}^{N}\bm{\phi}(x_n)t_n}_{=\bm{b}}
	= \bm{S} \bm{b} =: \bm{\xi}
	\end{align}
	to better understand how we can implement this in practice. By multiplying this expression with the inverse matrix $\bm{S}^{-1}$ from the left, we find
	\begin{align}
	b = \bm{S}^{-1}\bm{\xi} \, .
	\end{align}
	Since we do know how to explicitly construct $\bm{b}$ and $S^{-1}$, we can use this knowledge to find $\bm{\xi}$ by solving the linear system
	\begin{align}
	\bm{S}^{-1} \bm{\xi} = \bm{b}
	\end{align}
	for $\bm{\xi}$. Then we can construct the predicted mean $m(x)$ by computing
	\begin{align}
	m(x) = \beta\bm{\phi}(x)^T \cdot \bm{\xi}
	\end{align}
	
	\item To compute the predictive estimator for the variance we need to find
	\begin{align}
	s^2(x) = \beta^{-1} + \bm{\phi}(x)^T\underbrace{\bm{S}\bm{\phi}(x)}_{=:\bm{\chi}} \, ,
	\end{align}
	where we once again have the problem that we can explicitly construct $\bm{S}^{-1}$ but not $\bm{S}$ itself. Hence once again consider the expression
	\begin{align}
	\bm{S}\bm{\phi}(x) = \bm{\chi} \, .
	\end{align}
	Multiplying both sides by $\bm{S}^{-1}$ from the left yields
	\begin{align}
	\bm{\phi}(x) = \bm{S}^{-1}\bm{\chi} \, .
	\label{eq:linearSystemVariance}
	\end{align}
	To solve for the unknown $\bm{\chi}$ we hence need to solve the linear system as specified in equation \eqref{eq:linearSystemVariance} for $\bm{\chi}$. Thus we can find $s^2(x)$ by computing
	\begin{align}
	s^2(x) = \beta^{-1} + \bm{\phi}(x)^T\bm{\chi}
	\end{align}	
\end{itemize}

\subsection{Analytical pen and paper results for the mean and variance estimator for benchmarking numerical implementations}
The mean estimator using the Bayesian polynomial curve fitting ansatz is given by equation \eqref{eq:bayesCurveFitMean} as
\begin{align}
m(x) = \beta \bm{\phi}(x)^T \, \bm{S} \, \sum\displaylimits_{n=1}^{N}\bm{\phi}(x_n)t_n \, .
\end{align}
From the underlying theory we know how to explicitly construct $\bm{S}^{-1}$ as
\begin{align}
\bm{S}^{-1} = \alpha\mathds{1} + \beta \sum\displaylimits_{n=1}^{N} \phi(x_n)\phi(x_n)^T
\label{eq:Sinv}
\end{align}
Here we outline a pen and paper calculation using the following dummy data points. Assume that there are $N=2$ given training data points
\begin{align}
(x_1, t_1) = (0, 0) \quad \text{and} \quad (x_2, t_2) = (1, 1)\, .
\end{align}
For the sake of simplicity we further use $M=1$, \textit{i.e.} consider a $M=2$ dimensional problem and set the hyper parameters $\alpha = 1$ and $\beta =1$ both to unity.
Now we want to determine the mean $m(x)$ and the variance $s^2(x)$ for three arbitrarily chosen query points $x\in\{0,\, 0.5,\, 1\}$.

We start by constructing $\bm{S}^{-1}$ according to equation \eqref{eq:Sinv}, giving us
\begin{align}
\bm{S}^{-1} = \alpha \mathds{1} + \beta \sum\displaylimits_{n=1}^{N=2} \phi(x_n)\phi(x_n)^T \, , 
\end{align}
for which we need
\begin{align}
\phi(x_1) &= (x_1^0, \, x_1^1)^T = (1, \, 0)^T \\
\phi(x_2) &= (x_2^0, \, x_2^1)^T = (1, \, 1)^T \, .
\end{align}
With these expressions we find
\begin{align}
\bm{S}^{-1} = \begin{pmatrix}
1 & 0 \\
0 & 1
\end{pmatrix} + 
\begin{pmatrix}
1 \\ 0
\end{pmatrix} 
\cdot 
\begin{pmatrix}
1 & 0
\end{pmatrix}
+
\begin{pmatrix}
1 \\ 1
\end{pmatrix} 
\cdot 
\begin{pmatrix}
1 & 1
\end{pmatrix} = 
\begin{pmatrix}
1 & 0 \\
0 & 1
\end{pmatrix} +
\begin{pmatrix}
2 & 1 \\
1 & 1
\end{pmatrix} =
\begin{pmatrix}
3 & 1 \\
1 & 2
\end{pmatrix} \, .
\end{align}
Now we can compute the mean as
\begin{align}
m(x) = \beta\bm{\phi}(x)^T \cdot \bm{\xi} , \,
\end{align}
where $\bm{\xi}$ is the solution to linear system
\begin{align}
\bm{S}^{-1} \bm{\xi} = \bm{b} = \sum\displaylimits_{n=1}^{N=2}\phi(x_n) t_n = \begin{pmatrix}
1 \\ 1
\end{pmatrix}
\end{align}
Solving this 2d system for $\bm{\xi}$ gives
\begin{align}
\bm{\xi} = \begin{pmatrix}
1/5 \\ 2/5
\end{pmatrix}\, ,
\end{align}
where we used the explicit matrix inverse of $\bm{S}^{-1}$, which is
\begin{align}
(\bm{S}^{-1})^{-1} = \bm{S} = \dfrac{1}{5}\begin{pmatrix}
2 & -1 \\ -1 & 3
\end{pmatrix}\, .
\end{align}
Now we are set to compute the mean estimator for a given query value $x$, which we below show for the three arbitrarily chosen values.
\begin{align}
&m(x = 0) = \beta\bm{\phi}(x)^T \cdot \bm{\xi} = 1 \cdot (1 ,\, \, 0) \cdot \begin{pmatrix}
1/5 \\ 2/5
\end{pmatrix} = 1/5 = 0.2 \\
&m(x = 0.5) = \beta\bm{\phi}(x)^T \cdot \bm{\xi} = 1 \cdot (1 ,\, \, 0.5) \cdot \begin{pmatrix}
1/5 \\ 2/5
\end{pmatrix} = 2/5 = 0.4\\
&m(x = 1) = \beta\bm{\phi}(x)^T \cdot \bm{\xi} = 1 \cdot (1 ,\, \, 1) \cdot \begin{pmatrix}
1/5 \\ 2/5
\end{pmatrix} = 3/5 = 0.6
\end{align}

In order to calculate the variance estimator one needs to solve the linear system $\bm{S}^{-1} \bm{\chi} = \bm{\phi}(x)$, where the right hand side $\bm{\phi}(x)$ is now dependent of the query point $x$. Since we now the matrix $\bm{S}$ explicitly, we can simply compute $\bm{\chi} = \bm{S}\bm{\phi}(x)$ directly. For the three query points in this example we find
\begin{align}
\bm{\chi}(x = 0) = \begin{pmatrix}2/5 \\ -1/5\end{pmatrix} \, ,
\quad
\bm{\chi}(x = 0.5) = \begin{pmatrix}3/10 \\ 1/10\end{pmatrix} \, ,
\quad
\bm{\chi}(x = 1) = \begin{pmatrix}1/5 \\ 2/5\end{pmatrix} \, .
\end{align}
Having established expressions for $\bm{\chi}$ for all our test points, we can directly compute the variance estimator via
\begin{align}
s^2(x) = \dfrac{1}{\beta} + \bm{\phi}(x)^T \cdot \bm{\chi} \, ,
\end{align}
where we find
\begin{align}
&s^2(x = 0) = \dfrac{7}{5} = 1.4 \\ 
&s^2(x = 0.5) = \dfrac{27}{20} \\
&s^2(x = 1) = \dfrac{8}{5} = 1.6
\end{align}
Table \ref{tbl:benchmarkTable} summarizes these findings.

\rowcolors{1}{gray!25}{white}

\renewcommand{\arraystretch}{1.40}

\begin{center}
	\begin{table}
	\begin{tabularx}{8.0cm}{ccX}
		\hline
		query point $x$ & mean $m(x)$ & variance $s^2(x)$\\
		\hline
		$0$ & $1/5$ & $7/5$ \\  
		$0.5$ & $2/5$ & $27/20$ \\
		$1.0$ & $3/5$ & $8/5$ \\ 
		\hline  
	\end{tabularx}
	
	 \caption{Summary table.\label{tbl:benchmarkTable}}

\end{table}
\end{center}

\section{Design matrix}
For many statistical regression problems the term design matrix is useful and a reoccurring concept. Explain this here a little more...

\end{document}
